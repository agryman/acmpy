%! Author = arthurryman
%! Date = 2020-04-25

% Preamble
\documentclass[12pt]{article}

% Packages
\usepackage{../../../mathz/shared/preamble}
\usepackage{../../../mathz/articles/sets/sets}
\usepackage{../../../mathz/articles/groups/groups}
\usepackage{../../../mathz/articles/lie-groups-and-algebras/lie-groups-and-algebras}
\usepackage{so3-so5}

\usepackage{fancyhdr}
\usepackage{graphicx}
\graphicspath{ {./images/} }
\usepackage{listings}

\title{Notes on the Subgroup Inclusion $\SOR{3} \subset \SOR{5}$}
\author{
    Arthur Ryman\\
    \texttt{arthur.ryman@gmail.com}
}
\date{\today}


% Document
\begin{document}
    \maketitle

    \begin{abstract}
        This document describes the subgroup inclusion $\SOR{3} \subset \SOR{5}$
        that is used in the Algebraic Collective Model (ACM) of the atomic nucleus.
        All relevant concepts from group theory are formally defined here using Z notation.
    \end{abstract}

\section{Introduction}

The microscopic configuration space of an atomic nucleus contains the positions of each of its nucleons.
The collective motion configuration space consists of just the quadrupole moments of the mass distribution
of the nucleons.
ACM describes the atomic nucleus in terms of the quantized motions of the quadrupole moments. 

The rotation group of space $\SOR{3}$ acts on the microscopic configuration space.
These rotations induce corresponding $\SOR{5}$ rotations of quadrupole moment space.
This document describes the embedding of $\SOR{3}$ in $\SOR{5}$ that arises in ACM.

This document is motivated by the \texttt{acmpy} project which is developing a Python version
of a previously implemented Maple version of ACM.
The formal specification language, Z notation, is used here to help ensure the correctness of the Python version.
This document has been verified using the \fuzz\ type-checker for Z.

\end{document}
