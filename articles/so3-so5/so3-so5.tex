\documentclass{amsart}

\usepackage{parskip}

\usepackage{sets}
\usepackage{groups}
\usepackage{real-numbers}
\usepackage{lie-groups-and-algebras}
\usepackage{so3-so5}

\usepackage{preamble}

\begin{document}

\title{Quadrupole Moments and the Subgroup Inclusion $\SOR{3} \subset \SOR{5}$}
\author{Arthur Ryman}
\email[Arthur Ryman]{arthur.ryman@gmail.com}
\date{\today}

\begin{abstract}
This article describes quadrupole moments and the associated subgroup inclusion $\SOR{3} \subset \SOR{5}$
that is used in the \textit{Algebraic Collective Model} of the atomic nucleus.
All key concepts are formally specified here using \ZN\ in the hopes of improving the quality of software that
implement the model.
\end{abstract}

\maketitle

\tableofcontents

\section{Introduction}

This article describes part of the theory of the quantum mechanics of atomic nuclei.
The microscopic configuration space of an atomic nucleus consists of the positions of each of its nucleons.
Consider an atomic nucleus composed of $Z$ protons and $N$ neutrons.
The total number of nucleons $A$ is $Z + N$.
The dimension of the configuration space is $3A$.
The wave function of the nucleus is given by a complex-valued function of $3A$ coordinates.

In theory, we could model the wave function in a computer by storing its values on a grid of points in the configuration space.
If the grid contained $n$ distinct values for each coordinate then the entire grid would contain $n^{3A}$ points.
Clearly, as $A$ increases, the number of grid points quickly becomes larger than could be stored in any physical computer,

The Algebraic Collective Model (ACM) dramatically reduces the dimension of the configuration space to
just five by considering only the \textit{quadrupole moments} of the nucleon mass distribution.
The quadrupole moments are, in effect, summary statistics that can reveal the presence of \textit{collective motion}.

Furthermore, the ACM deals with Hamiltonians that are built up from operators in a certain Lie algebra.
This allows energy levels and transition amplitudes to be calculated algebraically.

Specifically, the quadrupole moments form a five-dimensional vector space that carries a natural representation
of the rotation group $\SOR{3}$ of three-dimensional Euclidean nucleon position space.
The usual inner product on nucleon position space induces a natural inner product on quadrupole moment space which is
preserved by the nucleon position space rotation group. This establishes a natural inclusion $\SOR{3} \subset \SOR{5}$.

This article is part of the \texttt{acmpy} project which is translating a previously developed Maple implementation of the ACM
into Python.
The formal specification language, \ZN, is used here to help ensure the correctness of the Python version.
This article has been checked using the \fuzz\ type checker for \ZN.

\section{Position Space}

For the purposes of this article, the configuration space of a single nucleon is the set of all positions in three-dimensional space.
Here we are ignoring spin and other quantum numbers.
Position space has a natural notion of distance defined by the usual dot product.
We refer to position space with this metric as \textit{Euclidean space}.

\subsection{Euclidean Space}

Let $EuclideanSpace$ denote the usual three-dimensional Euclidean position space.
A position is given by a triple of real numbers.

\begin{zed}
	EuclideanSpace == \R \cross \R \cross \R
\end{zed}

Let the notation $\Ethree$ denote $EuclideanSpace$.

\begin{zed}
	\Ethree == EuclideanSpace
\end{zed}

\subsection{Components}

Let $v \in \Ethree$ be a position in Euclidean space. 
Let the real numbers $x, y, z$ denote the components of $v = (x, y, z)$.

\begin{schema}{E3\_Components}
	v: \Ethree \\
	x, y, z: \R
\where
	v = (x, y, z)
\end{schema}

The space $\Ethree$ is a real three-dimensional vector space with addition and scalar multiplication
defined by componentwise addition and multiplication.
The points in $\Ethree$ will be referred to as \textit{position vectors} or, simply, as \textit{vectors} when the context is clear.

\subsection{Vector Addition}

Let $v_1$ and $v_2$ be vectors. Their sum $v$ is defined by componentwise addition.

\begin{schema}{E3\_VectorAddition}
	E3\_Components_1 \\
	E3\_Components_2 \\
	E3\_Components
\where
	v = (x_1 \addR x_2, y_1 \addR y_2, z_1 \addR z_2)
\end{schema}

Let the function $v = E3\_add(v_1, v_2)$ denote vector addition.

\begin{zed}
	E3\_add == \{~ E3\_VectorAddition @ (v_1, v_2) \mapsto v ~\}
\end{zed}

\begin{remark} 
Vector addition is a binary operation on Euclidean space.
\begin{zed}
	E3\_add \in \Ethree \cross \Ethree \fun \Ethree
\end{zed}
\end{remark}

Let the notation $v = v_1 \addEthree v_2$ denote vector addition in $\Ethree$.

\begin{zed}
	(\_ \addEthree \_) == E3\_add
\end{zed}

\subsubsection{Scalar Multiplication}

Let $s$ be a real scale factor and let $v_1$ be a vector. Their product $v$ is defined by componentwise multiplication.

\begin{schema}{E3\_ScalarMultiplication}
	s: \R \\
	E3\_Components_1 \\
	E3\_Components
\where
	v = (s \mulR x_1, s \mulR y_1, s \mulR z_1)
\end{schema}

Let the function $v = E3\_mul(s, v_1)$ denote scalar multiplication.

\begin{zed}
	E3\_mul == \{~ E3\_ScalarMultiplication @ (s, v_1) \mapsto v ~\}
\end{zed}

\begin{remark} 
Scalar multiplication is a total function.
\begin{zed}
	E3\_mul \in \R \cross \Ethree \fun \Ethree
\end{zed}
\end{remark}

Let the notation $v = s \mulEthree v_1$ denote scalar multiplication in $\Ethree$.

\begin{zed}
	(\_ \mulEthree \_) == E3\_mul
\end{zed}

\subsection{Linear Functions}

A real-valued function $f$ on $\Ethree$ is said to be a \textit{linear function} if it preserves vector addition and scalar product
in the following sense.

\begin{schema}{E3\_LinearFunction}
	f: \Ethree \fun \R
\where
	\forall v, w: \Ethree @ \\
	\t1	f(v \addEthree w) = (f~v) \addR (f~w)
\also
	\forall s: \R; v: \Ethree @ \\
	\t1	f(s \mulEthree v) = s \mulR (f~v)	
\end{schema}

The set of all linear functions on $\Ethree$ is called the \textit{dual space} of $\Ethree$.
Let $E3\_dual$ be the dual space.

\begin{zed}
	E3\_dual == \{~ E3\_LinearFunction @ f ~\}
\end{zed}

Let the notation $\dualEthree$ denote the dual space.

\begin{zed}
	\dualEthree == E3\_dual
\end{zed}

\subsection{Linear Transformations}

A function $T$ from $\Ethree$ to $\Ethree$ is a \textit{linear transformation} if it preserves vector addition and scalar product
in the following sense.

\begin{schema}{E3\_LinearTransformation}
	T: \Ethree \fun \Ethree
\where
	\forall v, w: \Ethree @ \\
	\t1	T(v \addEthree w) = (T~v) \addEthree (T~w)
\also
	\forall s: \R; v: \Ethree @ \\
	\t1	T(s \mulEthree v) = s \mulEthree (T~v)
\end{schema}

\subsection{The Identity Function}

Let $E3\_id$ denote the identity function from $\Ethree$ to $\Ethree$.

\begin{zed}
	E3\_id == \id \Ethree
\end{zed}

\begin{remark}
The identity function is a bijection.
\begin{zed}
	E3\_id \in \Ethree \bij \Ethree
\end{zed}
\end{remark}

Let the notation $\idEthree = E3\_id$ denote the identity function.

\begin{zed}
	\idEthree == E3\_id
\end{zed}

\subsection{Trace}

\subsection{Determinant}


\subsection{General Linear Transformations}

A linear transformation is said to be a \textit{general linear transformation} if it is a bijection.

\begin{schema}{E3\_GeneralLinearTransformation}
	E3\_LinearTransformation
\where
	T \in \Ethree \bij \Ethree
\end{schema}

\subsection{Special Linear Transformations}

A general linear transformation is said to be a \textit{special linear transformation} if its determinant is 1.

\subsection{The Dot Product}

Let $v_1$ and $v_2$ be vectors. Their dot product $g$ is the sum of their componentwise products.

\begin{schema}{E3\_DotProduct}
	E3\_Components_1 \\
	E3\_Components_2 \\
	g: \R
\where
	g = x_1 \mulR x_2 \addR y_1 \mulR y_2 \addR z_1 \mulR z_2
\end{schema}

Let the function $g = E3\_dot(v_1, v_2)$ denote the dot product.

\begin{zed}
	E3\_dot == \{~ E3\_DotProduct @ (v_1, v_2) \mapsto g ~\}
\end{zed}

\begin{remark}
\begin{zed}
	E3\_dot \in \Ethree \cross \Ethree \fun \R
\end{zed}
\end{remark}

Let the notation $g = v_1 \dotEthree v_2$ denote the dot product.

\begin{zed}
	(\_ \dotEthree \_) == E3\_dot
\end{zed}

\subsection{Orthogonal Transformations}

A linear transformation $T$ is \textit{orthogonal} if it preserves dot products.

\begin{schema}{E3\_OrthogonalTransformation}
	E3\_LinearTransformation
\where
	\forall v, w: \Ethree @ \\
	\t1	(T~v) \dotEthree (T~w) = v \dotEthree w
\end{schema}

\subsection{Special Orthogonal Transformations}

An orthogonal transformation is said to be \textit{special} if its determinant is $1$.



\end{document}
