%! Author = arthurryman
%! Date = 2020-04-25

% Preamble
\documentclass[12pt]{article}

% Packages
\usepackage{preamble}
\usepackage{fancyhdr}
\usepackage{graphicx}
\graphicspath{ {./images/} }
\usepackage{listings}

\usepackage{so3-so5}

\title{Notes on the Subgroup Inclusion $\SOR{3} \subset \SOR{5}$}
\author{
    Arthur Ryman\\
    \texttt{arthur.ryman@gmail.com}
}
\date{\today}


% Document
\begin{document}
    \maketitle

    \begin{abstract}
        This document describes the subgroup inclusion $\SOR{3} \subset \SOR{5}$
        that is used in the Algebraic Collective Model (ACM) of the atomic nucleus.
        All relevant concepts from group theory are formally defined here using Z notation.
    \end{abstract}

\section{Introduction}

The microscopic configuration space of an atomic nucleus contains the positions of each of its nucleons.
The collective motion configuration space consists of just the quadrupole moments of the mass distribution
of the nucleons.
ACM describes the atomic nucleus in terms of the quantized motions of the quadrupole moments. 

The rotation group of space $\SOR{3}$ acts on the microscopic configuration space.
These rotations induce corresponding $\SOR{5}$ rotations of quadrupole moment space.
This document describes the embedding of $\SOR{3}$ in $\SOR{5}$ that arises in ACM.

This document is motivated by the \texttt{acmpy} project which is developing a Python version
of a previously implemented Maple version of ACM.
The formal specification language, Z notation, is used here to help ensure the correctness of the Python version.
This document has been verified using the \fuzz\ type-checker for Z.

\section{Groups}

\subsection{$Semigroup$ and \zcmd{mulG}}

A {\em semigroup} is a set of elements endowed with an associative binary operation $\mulG$.
We often refer to this operation as the semigroup {\em multiplication}.

Let $Semigroup$ denote the set of all semigroups.

\begin{schema}{Semigroup}[\genT]
elements: \power \genT \\
\_ \mulG \_: \genT \cross \genT \pfun \genT
\where
(\_ \mulG \_) \in elements \cross elements \fun elements
\also
\forall x, y, z: elements @ (x \mulG y) \mulG z = x \mulG (y \mulG z)
\end{schema}

\subsubsection{$MapPreservesMultiplication$}

Let $A$ and $B$ be semigroups and let $f$ be a map of their underlying sets of elements.
The map $f$ is said to {\em preserve multiplication} 
if it maps products of elements to products of the mapped elements.

Let $MapPreservesMultiplication$ denote this situation.

\begin{schema}{MapPreservesMultiplication}[\genT, \genU]
f: \genT \pfun \genU \\
A: Semigroup[\genT] \\
B: Semigroup[\genU]
\where
f \in A.elements \fun B.elements
\also
\LET (\_ \mulG \_) == A.(\_ \mulG \_); \\
\t1	(\_ \timesG \_) == B.(\_ \mulG \_) @ \\
\t2		\forall x, y: A.elements @ \\
\t3			f(x \mulG y) = (f~x) \timesG (f~y)
\end{schema}

\subsubsection{\zcmd{HomSemigroup}}

A {\em semigroup homomorphism} from $A$ to $B$ is a map $f$ of the elements that preserves
multiplication.

Let $\HomSemigroup(A, B)$ denote the set of all semigroup homomorphisms from $A$ to $B$.

\begin{gendef}[\genT, \genU]
\HomSemigroup: Semigroup[\genT] \cross Semigroup[\genU] \fun \power (\genT \pfun \genU)
\where
\HomSemigroup = \\
\t1	(\lambda A: Semigroup[\genT]; B: Semigroup[\genU] @ \\
\t2		\{~ f: A.elements \fun B.elements | \\
\t3			MapPreservesMultiplication[\genT, \genU] ~\})
\end{gendef}

\begin{remark}
The identity mapping is a semigroup homomorphism.
\end{remark}

\begin{remark}
The composition of two semigroup homomorphisms is another semigroup homomorphism.
\end{remark}

\subsection{$Monoid$ and \zcmd{oneG}}

A {\em monoid} is a semigroup that has a left and right identity element $\oneG$.

Let $Monoid$ denote the set of all monoids.

\begin{schema}{Monoid}[\genT]
Semigroup[\genT] \\
\oneG: \genT
\where
\oneG \in elements
\also
\forall x: elements @ \oneG \mulG x = x = x \mulG \oneG
\end{schema}

\subsubsection{\zcmd{MonoidSemigroup}}

Given a monoid, we can forget its identity element and obtain a semigroup.

Let $\MonoidSemigroup$ denote the function that maps a monoid to its underlying semigroup.

\begin{gendef}[\genT]
\MonoidSemigroup: Monoid[\genT] \fun Semigroup[\genT]
\where
\MonoidSemigroup = \\
\t1	(\lambda Monoid[\genT] @ \theta Semigroup)
\end{gendef}

\begin{remark}
If a semigroup has an identity element then it is unique.
This means that forgetting the identity element defines an injection of the set of monoids into the set of
semigroups.
$$
\MonoidSemigroup[\genT] \in Monoid[\genT] \inj Semigroup[\genT]
$$
\end{remark}

\begin{proof}
Suppose $\oneG$ and $\oneG'$ are identity elements.
\begin{argue}
\oneG \\
\t1	= \oneG \mulG \oneG'	& $\oneG'$ is a right identity element \\
\t1	= \oneG'				& $\oneG$ is a left identity element
\end{argue}
\end{proof}

\subsubsection{$MapPreservesIdentity$}

Let $A$ and $B$ be monoids and let $f$ map the elements of $A$ to the elements of $B$.
The map $f$ is said to {\em preserve the identity element} if it maps the identity element of $A$
to the identity element of $B$.

Let $MapPreservesIdentity$ denote this situation.

\begin{schema}{MapPreservesIdentity}[\genT, \genU]
f: \genT \pfun \genU \\
A: Monoid[\genT] \\
B: Monoid[\genU]
\where
f \in A.elements \fun B.elements
\also
f(A.\oneG) = B.\oneG
\end{schema}

\subsubsection{\zcmd{HomMonoid}}

A {\em monoid homomorphism} from $A$ to $B$ is a homomorphism $f$ of the underlying semigroups
that preserves identity.

Let $\HomMonoid(A, B)$ denote the set of all monoid homomorphisms from $A$ to $B$.

\begin{gendef}[\genT, \genU]
\HomMonoid: Monoid[\genT] \cross Monoid[\genU] \fun \power (\genT \pfun \genU)
\where
\HomMonoid = \\
\t1	(\lambda A: Monoid[\genT]; B: Monoid[\genU] @ \\
\t2		\{~ f: \HomSemigroup(\MonoidSemigroup A, \MonoidSemigroup B) | \\
\t3			MapPreservesIdentity[\genT, \genU] ~\})
\end{gendef}

\begin{remark}
The identity mapping is a monoid homomorphism.
\end{remark}

\begin{remark}
The composition of two monoid homomorphisms is another monoid homomorphism.
\end{remark}

\subsection{$Group$ and \zcmd{invG}}

A {\em group} is a monoid for which every element $x$ has an inverse $x \invG$.

Let $Group$ denote the set of all groups.

\begin{schema}{Group}[G]
Monoid[G] \\
\_ \invG: G \pfun G
\where
(\_ \invG) \in elements \fun elements
\also
\forall x: elements @ x \mulG x \invG = \oneG = x \invG \mulG x
\end{schema}

\subsubsection{\zcmd{GroupMonoid}}

Given a group, we can forget its inverse operation and obtain a monoid.

\begin{gendef}[\genT]
\GroupMonoid: Group[\genT] \fun Monoid[\genT]
\where
\GroupMonoid = \\
\t1	(\lambda Group[\genT] @ \theta Monoid)
\end{gendef}

\begin{remark}
If a monoid has an inverse operation then it is unique.
This means that forgetting the inverse operation defines an injection of the set of groups into the set of
monoids.
$$
\GroupMonoid[\genT] \in Group[\genT] \inj Monoid[\genT]
$$
\end{remark}

\begin{proof}
Let $x$ be any element.
Suppose $x \invG$ and $x \daggerG$ are inverses of $x$.
\begin{argue}
x\daggerG \\
\t1	= x\daggerG \mulG \oneG				& $\oneG$ is an identity element \\
\t1	= x\daggerG \mulG (x \mulG x \invG)		& $x \invG$ is an inverse \\
\t1	= (x\daggerG \mulG x) \mulG x \invG		& associativity \\
\t1	= \oneG \mulG x \invG				& $x \daggerG$ is an inverse \\
\t1	= x \invG							& $\oneG$ is an identity element
\end{argue}
\end{proof}

\subsubsection{$MapPreservesInverse$}

Let $A$ and $B$ be groups and let $f$ map the elements of $A$ to the elements of $B$.
The map $f$ is said to {\em preserve the inverses} if it maps the inverses of element of $A$
to the inverses of the corresponding elements of $B$.

Let $MapPreservesInverse$ denote this situation.

\begin{schema}{MapPreservesInverse}[\genT, \genU]
f: \genT \pfun \genU \\
A: Group[\genT] \\
B: Group[\genU]
\where
f \in A.elements \fun B.elements
\also
\LET (\_ \invG) == A.(\_ \invG); \\
\t1	(\_ \daggerG) == B.(\_ \invG) @ \\
\t2		\forall x: A.elements @ \\
\t3			f(x \invG) = (f~x) \daggerG
\end{schema}

\subsubsection{\zcmd{HomGroup}}

A {\em group homomorphism} from $A$ to $B$ is a homomorphism $f$ of the underlying monoids
that preserves inverses.

Let $\HomGroup(A, B)$ denote the set of all group homomorphisms from $A$ to $B$.

\begin{gendef}[\genT, \genU]
\HomGroup: Group[\genT] \cross Group[\genU] \fun \power (\genT \pfun \genU)
\where
\HomGroup = \\
\t1	(\lambda A: Group[\genT]; B: Group[\genU] @ \\
\t2		\{~ f: \HomMonoid(\GroupMonoid A, \GroupMonoid B) | \\
\t3			MapPreservesInverse[\genT, \genU] ~\})
\end{gendef}

\begin{remark}
The identity mapping is a group homomorphism.
\end{remark}

\begin{remark}
The composition of two group homomorphisms is another group homomorphism.
\end{remark}

\subsection{$AbelianGroup$}

An Abelian group is a group in which the binary operation is commutative.
It is conventional to write the group operations as additive instead of multiplicative in some situations.
We therefore introduce additive synonyms for the group components.

Let $AbelianGroup$ denote the set of all Abelian groups.

\begin{schema}{AbelianGroup}[G]
Group[G] \\
\_ \addG \_: G \cross G \pfun G \\
\zeroG: G \\
\negG: G \pfun G
\where
\forall x, y: elements @ x \mulG y = y \mulG x
\also
(\_ \addG \_) = (\_ \mulG \_)
\also
\zeroG = \oneG
\also
\negG = (\_ \invG)
\end{schema}

\begin{itemize}
\item The group binary operation is commutative.
\item Addition is a synonym for the group binary operation.
\item Zero is a synonym for the identity element.
\item Negative is a synonym for inverse.
\end{itemize}

\section{Real Numbers}

Z notation does not predefine the set of real numbers, so we define it here.

\subsection{\zcmd{RR}}

Let $\RR$ denote the set of real numbers.
We define it to be simply a given set.
We'll add further axioms as needed below.

\begin{zed}
[\RR]
\end{zed}

\subsection{\zcmd{addR}, \zcmd{zeroR}, \zcmd{negR}, and \zcmd{subR}}

Let $x$ and $y$ be real numbers.
Let $x \addR y$ denote addition, 
let $\zeroR$ denote zero,
let $\negR x$ denote negative,
and let $x \subR y$ denote subtraction.

Although these real number objects are displayed the same as the corresponding integer objects, 
they represent distinct mathematical objects.
This distinction is apparent to the \fuzz\ type-checker and should not cause confusion to the human reader
because the underlying types of objects will, as a rule, be clear from the context.
Visually distinct symbols will be used in cases where confusion is possible.

The real numbers form an Abelian group under addition.

\begin{axdef}
\_ \addR \_: \RR \cross \RR \fun \RR \\
\zeroR: \RR \\
\negR: \RR \fun \RR
\where
\exists_1 A: AbelianGroup[\RR] @ \\
\t1	A.elements = \RR \land \\
\t1	A.(\_ \addG \_) = (\_ \addR \_) \land \\
\t1	A.\zeroG = \zeroR \land \\
\t1	A.\negG = \negR
\end{axdef}

Subtraction is defined in terms of addition and negative.

\begin{axdef}
\_ \subR \_: \RR \cross \RR \fun \RR
\where
\forall x, y: \RR @ x \subR y = x \addR (\negR y)
\end{axdef}

\subsection{\zcmd{Rnz}}

Let $\Rnz$ denote the set of non-zero real numbers.

\begin{zed}
\Rnz == \RR \setminus \{ \zeroR \}
\end{zed}

\subsection{\zcmd{mulR}}

Let $x$ and $y$ be real numbers.
Let $x \mulR y$ denote multiplication.

\begin{axdef}
\_ \mulR \_: \RR \cross \RR \fun \RR
\end{axdef}

\subsection{\zcmd{mulRnz}, \zcmd{oneR}, \zcmd{invRnz}, and \zcmd{divR}}

Let $(\_ \mulRnz \_)$ denote the restriction of $(\_ \mulR \_)$ to $\Rnz$.

\begin{axdef}
\_ \mulRnz \_: \Rnz \cross \Rnz \fun \Rnz
\where
(\_ \mulRnz \_) = (\lambda x, y: \Rnz @ x \mulR y)
\end{axdef}

Let $x$ be real number and let $y$ be a non-zero real number.
let $\oneR$ denote one,
let $y \invRnz$ denote inverse,
and let $x \divR y$ denote division.

\begin{axdef}
\oneR: \Rnz \\
\_ \invRnz: \Rnz \fun \Rnz
\end{axdef}

The non-negative real numbers form an Abelian group under multiplication.

\begin{zed}
\exists_1 A: AbelianGroup[\Rnz] @ \\
\t1	A.elements = \Rnz \land \\
\t1	A.(\_ \mulG \_) = (\_ \mulRnz \_) \land \\
\t1	A.\oneG = \oneR \land \\
\t1	A.(\_ \invG) = (\_ \invRnz)
\end{zed}

Division is defined in terms of multiplicative inverse.

\begin{axdef}
\_ \divR \_: \RR \cross \Rnz \fun \RR
\where
\forall x: \RR; y: \Rnz @ x \divR y = x \mulR (y \invRnz)
\end{axdef}

Addition is distributive over multiplication.

\begin{zed}
\forall x, y, z: \RR @ (x \addR y) \mulR z = x \mulR z \addR y \mulR z
\end{zed}

\section{Real Vector Spaces}

\subsection{\zcmd{addV}, \zcmd{zeroV}, \zcmd{negV}, and \zcmd{mulS}}

Let $v$ and $w$ denote vectors and let $x$ denote a real number.
Let $v \addV w$ denote vector addition, 
let $\zeroV$ denote the zero vector, 
let $\negV v$ denote the negative vector,
and let $x \mulS v$ denote scalar multiplication.

Let $RealVectorSpace$ denote the set of all real vector spaces.

\begin{schema}{RealVectorSpace}[V]
vectors: \power V \\
\_ \addV \_: V \cross V \pfun V \\
\zeroV: V \\
\negV: V \pfun V \\
\_ \mulS \_: \RR \cross V \pfun V
\where
\exists_1 A: AbelianGroup[V] @ \\
\t1	A.elements = vectors \land \\
\t1	A.(\_ \addG \_) = (\_ \addV \_) \land \\
\t1	A.\zeroG = \zeroV \land \\
\t1	A.\negG = \negV
\also
(\_ \mulS \_) \in \RR \cross vectors \fun vectors
\also
\forall v: vectors @ \zeroR \mulS v = \zeroV
\also
\forall v: vectors @ \oneR \mulS v = v
\also
\forall x, y: \RR; v: vectors @ (x \mulR y) \mulS v = x \mulS (y \mulS v)
\also
\forall x, y: \RR; v: vectors @ (x \addR y) \mulS v = x \mulS v \addV y \mulS v
\also
\forall x: \RR; v, w: vectors @ x \mulS (v \addV w) = x \mulS v \addV x \mulS w
\end{schema}

\begin{itemize}
\item Vectors form an Abelian group under addition.
\item Multiplying a vector by a scalar gives a vector.
\item Multiplying by $\zeroR$ gives the zero vector.
\item Multiplying by $\oneR$ gives the same vector.
\item Scalar multiplication is associative.
\item Scalar addition distributes over scalar multiplication.
\item Vector addition distributes over scalar multiplication.
\end{itemize}

\subsection{Linear Transformations}

Let $X$ and $Y$ be vector spaces.
A {\em linear transformation} from $X$ to $Y$ is a homomorphism of the underlying Abelian groups
that preserves scalar multiplication.

\end{document}
